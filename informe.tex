\documentclass[11pt]{article}
\usepackage[a4paper, margin=1.5cm]{geometry}
\usepackage[utf8]{inputenc}
\usepackage{babel}
\usepackage[spanish]{layout}
\usepackage[article]{ragged2e}
\usepackage{textcomp}
\usepackage{amsmath}
\usepackage{amssymb}
\usepackage{amsfonts}
\usepackage{proof}
\usepackage{enumerate}
\usepackage{graphicx}
\usepackage{multirow}

\setlength{\parindent}{0pt}

\title{
    Ontología del Mundo Mágico de Harry Potter}
\author{Mellino, Natalia \and Farizano, Juan Ignacio}
\date{}

\begin{document}
\maketitle

\noindent\rule{\textwidth}{1pt}

\section{Descripción}

\-\ Nos interesa estudiar las diferentes escuelas del mundo mágico de Harry Potter.
Cada escuela puede estar divida en 4 casas o en ninguna, donde los alumnos serán 
repartidas en ellas. Cada casa tiene un jefe y dos prefectos, un hombre y una 
mujer. Un jefe es siempre un profesor de la escuela, mientras que los roles de 
prefectos los ocupan dos estudiantes de esa misma casa que estén cursando desde 
quinto año en adelante.

\-\ Los personajes en este mundo mágico las podemos categorizar según el nivel de 
magia que posean: Magos o Brujas, Squibs o Muggles. Los Magos o Brujas son 
aquellos que poseen magia, los Squibs son los que no tienen magia pero provienen 
de padres magos mientras que los Muggles no poseen nada de magia. En este mundo, 
sólo los Magos o Brujas pueden ser alumnos o profesores de una escuela.

\-\ En una escuela los estudiantes deben atender a clases que se dictan durante todo 
el año. Las clases van variando a medida que van avanzando de año. Un alumno se 
gradua al finalizar séptimo año.
Durante el año, los alumnos tiene diferentes actividades recreativas, siendo las 
más importantes:
    \begin{itemize}
        \item La Copa de las Casas que es un galardón otorgado al final de cada 
        año a la casa que reunió más puntos durante el transcurso del año 
        mediante diferentes actividades.
        \item La principal disciplina es el famoso deporte mágico Quidditch, 
        cada casa tiene su propio equipo, el cual esta conformado por siete 
        jugadores y tiene un capitán que puede ser cualquiera de los integrantes 
        del equipo. En este deporte, compiten en un campeonato las casas de las 
        escuelas entre sí. 
    \end{itemize}

\section{Modelado de la ontología}

\subsection{Jerarquía de clases}

fari fix this :(

\begin{itemize}
    \item Escuela
    \item Casa
    \item Persona
          \begin{itemize}
              \item Mago/Bruja
                    \begin{itemize}
                      \item Alumno Docente 
                    \end{itemize}
              \item Squib
              \item Muggle
          \end{itemize}
    \item Clase
    \item Equipo de Quidditch
\end{itemize}

\subsection{Propiedades de cada clase}

\begin{itemize}
    \item Escuela
          \begin{itemize}
              \item Nombre
              \item Ubicación
              \item Año de fundación
              \item Lema 
          \end{itemize}
    \item Casa
          \begin{itemize}
              \item Nombre
              \item Cantidad de Copas de las Casas ganadas 
          \end{itemize}
    \item Persona
          \begin{itemize}
              \item Nombre 
              \item Edad
          \end{itemize}
    \item Clase
          \begin{itemize}
              \item Nombre
              \item Año 
          \end{itemize}
    \item Equipo de Quidditch
          \begin{itemize}
              \item Número de campeonatos ganados 
          \end{itemize}
\end{itemize}

\subsection{Relaciones}

\begin{itemize}
    \item Alumno \emph{estudiaEn} Escuela
    \item Persona \emph{trabajaEn} Escuela
    \item Alumno \emph{perteneceA} Casa
    \item Alumno \emph{estudia} Clase
    \item Docente \emph{enseña} Clase
    \item Equipo de Quidditch \emph{perteneceA} Casa
    \item Alumno \emph{juegaPara} Equipo de Quidditch
    \item Docente \emph{Dirige} Escuela
    \item Alumno \emph{esPrefectoDe} Casa
    \item Mago/Bruja \emph{esFundadorDe} Casa
    \item Casa \emph{esCasaDe} Escuela 
\end{itemize}

En Protégé se han realizado también las \textbf{relaciones inversas} para la
mayoría de las relaciones presentadas en este informe.

\section{Queries}

aca pondria mis queries SI TUVIERA UNAAAAAAAAAAAAAAAAAAAA

\section{Conclusión}

la conclusion es que no hay conclusion xd

\end{document}